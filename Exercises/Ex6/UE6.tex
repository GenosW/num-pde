\documentclass[11pt,a4paper]{article}
\usepackage[left=2.5cm,right=2cm, bottom=2cm]{geometry}
\usepackage[utf8]{inputenc}
\usepackage{amsmath}
\usepackage{amsfonts}
\usepackage{amssymb}
\usepackage{amsfonts}
\usepackage{amsmath}
\usepackage{graphicx}
\usepackage{subfigure}
\usepackage{color}
\usepackage{abstract}
\usepackage{float}
\usepackage[toc,page]{appendix}
\usepackage{hyperref}
\usepackage{fancyhdr}

\pagestyle{fancy}
\fancyhf{}
\rhead{\today}
\lhead{\bfseries Alexander Leitner 01525882}
\rfoot{}



\begin{document}
\begin{center}
    \fontsize{24pt}{10pt}\selectfont
    \textsc{\textbf{NUMPDE Exercise 6}}
\end{center}
\section{Example 6.1}
Proof that:
\begin{align*}
Q_T(v) = \int_T v\,dx\,\,\,\, \forall v \in \mathbb{P}^2(T)
\end{align*}
First we define a $\hat{v}(\hat{x},\hat{y})$ Element and integral over it. 
\begin{align*}
\hat{v}(\hat{x},\hat{y}) = c_0+c_1\hat{x}+c_2\hat{y}+c_3\hat{x}^2+c_4\hat{y}^2+c_5\hat{x}\hat{y}
\end{align*}
\begin{align*}
\int_{\hat{x}=0}^1\int_{\hat{y}=0}^{1-\hat{x}}\hat{v}d\hat{x} = \frac{c_0}{2}+\frac{c_1+c_2}{6}+\frac{c_3+c_4}{12}+\frac{c_5}{24}
\end{align*}
Now we can use the midpoint rule to calculate the area $\in V$.
\begin{align*}
Q_{\hat{T}}(v) := \frac{|\hat{T}|}{3}\big[v(\hat{x}_1)v(\hat{x}_2)v(\hat{x}_3)\big]
\end{align*}
Also the are of $|\hat{T}| = \frac{1}{2}|T|$ and  \\\\
$\hat{v}(\hat{x_1}) = \hat{v}(\frac{1}{2},0) = c_0 + \frac{c_1}{2} + \frac{c_3}{4}$\\
$\hat{v}(\hat{x_2}) = \hat{v}(\frac{1}{2},\frac{1}{2}) = c_0+\frac{c_1}{2}+\frac{c_3}{4}$\\
$\hat{v}(\hat{x_3}) = \hat{v}(0,\frac{1}{2}) = c_0+\frac{c_1}{2} + \frac{c_2}{2}+\frac{c_3}{4}+\frac{c_4}{4}+\frac{c_5}{4}$\\\\
$Q_{\hat{T}}(\hat{v}) = \frac{c_0}{2}+\frac{c_1+c_2}{6}+\frac{c_3+c_4}{12}+\frac{c_5}{24}$\\
We showed finally that $Q_{\hat{T}}(\hat{v}) = \int_{\hat{T}}\hat{v}$ and we know that lagrange FE are equivalent. We define $F: \hat{T} \to T$ as an affine linear mapping.
\begin{align*}
\int_{T}v(x)\, dx  = \int_T\hat{v}\, \circ\, F^{-1}(x)\,dx = \int_{\hat{T}}\Big(\hat{v}\,\circ F^{-1}\Big)\circ\,F(\hat{x})|\det DF|\,d\hat{x}=|\det DF|\int_{\hat{T}}\hat{v}(\hat{x})\,d\hat{x}
\end{align*}
At last we only have to show that $\int_{\hat{T}}\hat{v}(\hat{x})\,d\hat{x} = Q_{\hat{T}(\hat{v})}$
\begin{align*}
& |\det DF|Q_{\hat{T}(\hat{v})} = |\det DF|\frac{|\hat{T}|}{3}\Big(\hat{v}(\hat{x_1})+\hat{v}(\hat{x_2})+\hat{v}(\hat{x_3})\Big)\\ & =
\frac{|\hat{T}|}{3} \Bigg((v\circ F)(F^{-1}(x_1))+(v\circ F)(F^{-1}(x_2))+(v\circ F)(F^{-1}(x_3))\Bigg) \\ & = \frac{|\hat{T}|}{3}\Big(v(x_1)+v(x_2)+v(x_3)\Big)
\end{align*}
\newpage
\section{Example 6.2}
This proof stats with Theorem 69 and we start with where $Q_h(v) = \sum_{T \in \mathbb{T}}Q_T(v) =\sum_{T \in \mathbb{T}} \int_{T}v \, dx$:
\begin{align*}
&|\int_{\Omega} v \, dx -Q_h(v)|^2 = |\int_{\Omega}v\,dx-\sum_{T \in T_h}\int_TI_Tv\,dx|^2 = |\sum_{T \in T_h}\int_T(v-I_Tv)dx|^2 \\ & = \sum_{T \in \mathbb{T}}|\int(id-I_T)v_T\, dx |^2 \leq \sum_{T \in T_h}||\int_T(id-I_T)v\,dx||_{L^2(T)}^2
\end{align*}
Now using (4.2)
\begin{align*}
\sum_{T \in T_h}\det(B)||\int(id-I_T)v\,dx||^2_{L_2(\hat{T})}
\end{align*}
and after the transformation to the reference triangle where $I_T = I_{\hat{T}}$
\begin{align*}
\sum_{T \in T_h}\det(B)||\int \det(B)(id-I_{\hat{T}})(v\,\circ\,F_T)\,d\hat{x}||^2_{L_2(\hat{T})}
\end{align*}
After that we apply the Bramble Hilbert lemma. The Quatrature rule is exact for polynominals for decreas 2 as we seen in 6.1 $\Rightarrow \int(id-I_{\hat{T}})q\,dx = Lq = 0$ for $q \in \mathbb{P}^2(\hat{T})$\\Theorem 54: $||Lu||_a \leq $$ ||u_{H^k}$
\begin{align*}
 \leq \sum_{T \in \mathbb{T}} \det(B)|v\,\circ\,F_T|^2_{H^3(\hat{T})}
\end{align*}
\begin{align*}
\preceq \sum_{T \in \mathbb{T}}\det(B)\det(B)^{-1}||B||^6|v|^2_{H^3(T)}
\end{align*}
In the end we use the quasi-uniform $||B||^6 \backsimeq h^6$ propperty:
\begin{align*}
||\int_{\Omega}v\,dx-Q_h(v)||_{L_2(\Omega)}\preceq h^3|v|_{H^3(\Omega)}
\end{align*}
\section{Example 6.3}
We first proof that $v_h \in V^k_{h,0}$ there holds: $||e'_h||^2_{L^2(\Omega)} = (e'_h,e'_h-v'_h)_{L^2(\Omega)}$
\begin{align*}
\int u'v'\,d\Omega = A(u,v) \,\,\,\,\,\,\,\, \int f\,v\,d\Omega = f(v)
\end{align*}
\begin{align*}
& A(e_h,e_h-v_h) = A(u-u_h,u-u_h)-A(u-u_h,v_h)\\
& = A(u-u_h,u-u_h)-A(u,v_h)+A(u_h,v_h)\\
& = A(u-u_h,u-u_h)-f(v_h)+f(v_h) = A(e_h,e_h) \\
& = ||e'_h||^2_{L^2(\Omega)}
\end{align*}
By applay the integration by parts:
\begin{align*}
\int e'_h(e'_h-v'_h)\,dT = \frac{d}{dx}\int e'_h(e_h-v_h)\,dT-\int e''_h(e_h-v_h)\,dT
=e''_h = f + u''_h
\end{align*}
where $\frac{d}{dx}\int e'_h(e_h-v_h)\,dT = 0$
And so:
\begin{align*}
||e'_h||_{L^2(\Omega)} & = \sum_{T \in T_h}||e'_h||^2_{L^2(T)} = \sum_{T \in T_h}(e'_h,e'_h-v'_h)_{L^2(T)}\\
& = \sum_{T \in T_h} (-e''_h,e_h-v_h)_{L^2(T)}\\
& \leq \sum_{T \in T_h}(-e''_h,e_h-I_he_h)_{L^2(T)}\\
& = \sum_{T \in T_h}(f+u''_h,e_h-I_he_h)_{L^2(T)}
\end{align*}
For the second proof we start:
\begin{align*}
&||e'_h||^2_{L^2(\Omega)} = |u-u_h|_{H_1(\Omega)}\\
& e'_h-I_he'_h = u'-u'_h-I_hu'+I_hu'_h = u'-I_hu'
\end{align*}
\begin{align*}
& ||e'_h||^2_{L^2(\Omega)} = \sum_{T \in T_h}||e'_h||^2_{L^2(T)} = \sum_{T \in T_h}(u'-u'_h,u'-u_h')_{L^2(T)}\\
& \leq \sum_{T \in T_h}(u'-I_hu',u'-I_hu')_{L^2(T)} = \sum_{T \in T_h}(e'_h-I_he'_h,e'_h-I_he'_h)_{L^2(T)}\\
& c \sum_{T \in T_h} h_T^2|e'_h|^2_{H_1} = c \sum_{T \in T_h} h_T^2||-(f+u_h'')^2_{L^2} = c \sum_{T \in T_h}h_T^2||f+u''_h||^2_{L^2}\\
& |u-u_h|_{H^1(\Omega)} \leq \sum_{T \in T_h} h_T^2||f+u_h''||^2_{L^2}
\end{align*}


\begin{align*}
y = a_0+a_1x
\end{align*}

\begin{align*}
J = \frac{1}{n}\sum_{i = 1}^n(\hat{y}_i-y_i)^2
\end{align*}

\begin{align*}
J_L = \sum_{i = 1}^n(\hat{y}_i-y_i)^2+\alpha \sum_{j = 1}^m|\omega_j|
\end{align*}

\end{document}