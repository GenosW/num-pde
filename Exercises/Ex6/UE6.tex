\documentclass[11pt,a4paper]{article}
\usepackage[left=2.5cm,right=2cm, bottom=2cm]{geometry}
\usepackage[utf8]{inputenc}
\usepackage{amsmath}
\usepackage{amsfonts}
\usepackage{amssymb}
\usepackage{amsfonts}
\usepackage{amsmath}
\usepackage{graphicx}
\usepackage{subfigure}
\usepackage{color}
\usepackage{abstract}
\usepackage{float}
\usepackage[toc,page]{appendix}
\usepackage{hyperref}
\usepackage{fancyhdr}

\pagestyle{fancy}
\fancyhf{}
\rhead{\today}
\lhead{\bfseries Alexander Leitner 01525882}
\rfoot{}



\begin{document}
\begin{center}
    \fontsize{24pt}{10pt}\selectfont
    \textsc{\textbf{NUMPDE Exercise 6}}
\end{center}
\section{Example 6.1}
Proof that:
\begin{align*}
Q_T(v) = \int_T v\,dx\,\,\,\, \forall v \in \mathbb{P}^2(T)
\end{align*}
Forst we difin a $\hat{v}(\hat{x},\hat{y})$ Element and integrat over it. 
\begin{align*}
\hat{v}(\hat{x},\hat{y}) = c_0+c_1\hat{x}+c_2\hat{y}+c_3\hat{x}^2+c_4\hat{y}^2+c_5\hat{x}\hat{y}
\end{align*}
\begin{align*}
\int_{\hat{x}=0}^1\int_{\hat{y}=0}^{1-\hat{x}}\hat{v}d\hat{x} = \frac{c_0}{2}+\frac{c_1+c_2}{6}+\frac{c_3+c_4}{12}+\frac{c_5}{24}
\end{align*}
Now we kan use the midpoint rule to calculate the area $\in V$.
\begin{align*}
Q_{\hat{T}}(v) := \frac{|\hat{T}|}{3}\big[v(\hat{x}_1)v(\hat{x}_2)v(\hat{x}_3)\big]
\end{align*}
Also the are of $|\hat{T}| = \frac{1}{2}|T|$ and $\hat{v}(\hat{x_1}) = c_0+\frac{c_1}{2}+\frac{c_3}{4}$,
$\hat{v}(\hat{x_2}) =c_0+\frac{c_1}{2}+\frac{c_3}{4}$,
$\hat{v}(\hat{x_3}) = c_0+\frac{c_1}{2}+\frac{c_2}{2}+\frac{c_3}{4}+\frac{c_4}{4}+\frac{c_5}{4}$\\
$Q_{\hat{T}}(\hat{v}) = \frac{c_0}{2}+\frac{c_1+c_2}{6}+\frac{c_3+c_4}{12}+\frac{c_5}{24}$\\
We sowed finally that $Q_{\hat{T}}(\hat{v}) = \int_{\hat{T}}\hat{v}$ and we know that lagrange FR are equivalent. We define $F: \hat{T} \to T$ as an affine linear mapping.
\begin{align*}
\int_{T}v(x)\, dx  = \int_T\hat{v}\, \circ\, F^{-1}(x)\,dx = \int_{\hat{T}}\Big(\hat{v}\,\circ F^{-1}\Big)\circ\,F(\hat{x})|\det DF|\,d\hat{x}=|\det DF|\int_{\hat{T}}\hat{v}(\hat{x})\,d\hat{x}
\end{align*}
At last we only have to show that $\int_{\hat{T}}\hat{v}(\hat{x})\,d\hat{x} = Q_{\hat{T}(\hat{v})}$
\begin{align*}
|\det DF|Q_{\hat{T}(\hat{v})} = |\det DF|\frac{|\hat{T}|}{3}\Big(\hat{v}(\hat{x_1})+\hat{v}(\hat{x_2})+\hat{v}(\hat{x_3})\Big)\\ =
\frac{|\hat{T}|}{3} \Bigg((v\circ F)(F^{-1}(x_1))+(v\circ F)(F^{-1}(x_2))+(v\circ F)(F^{-1}(x_3))\Bigg) = \frac{|\hat{T}|}{3}\Big(v(x_1)+v(x_2)+v(x_3)\Big)
\end{align*}
\section{Example 6.2}
This proof stats with Theorem 69 and we start with where $Q_h(v) = \sum_{T \in \mathbb{T}}Q_T(v) =\sum_{T \in \mathbb{T}} \int_{T}v \, dx$:
\begin{align*}
||\int_{\Omega} v \, dx -Q_h(v)||_{L_2(T)} = \sum_{T \in \mathbb{T}}||\int(id-I_T)v_T\, dx ||^2_{L_2(T)} 
\end{align*}
Now using (4.2)
\begin{align*}
\sum_{T \in \mathbb{T}}\det(B)||\int(id-I_T)v_T\,dx||^2_{L_2(\hat{T})}
\end{align*}
and after the transformation to the reference triangle:
\begin{align*}
\sum_{T \in \mathbb{T}}\det(B)||\int \det(B)(id-I_{\hat{T}})(v_T\,\circ\,F_T)\,d\hat{x}||^2_{L_2(\hat{T})}
\end{align*}
After that we apply the Bramble Hilbert lemma. The Quatrature rule is exact for polynominals for decreas 2 as we seen in 6.1 $\Rightarrow \int(id-I_{\hat{T}})q\,dx = 0$ for $q \in \mathbb{P}^2(\hat{T})$
\begin{align*}
||\int_{\Omega} v \, dx -Q_h(v)||_{L_2(T)} \leq \sum_{T \in \mathbb{T}} \det(B)|v_T\,\circ\,F_T|^2_{H^3(\hat{T})}
\end{align*}
\begin{align*}
\preceq \sum_{T \in \mathbb{T}}\det(B)\det(B)^{-1}||B||^6|v_T|^2_{H^3(T)}
\end{align*}
In the end we use the quasi-uniform $||B||^6 \backsimeq h^6$ propperty:
\begin{align*}
||\int_{\Omega}v\,dx-Q_h(v)||_{L_2(\Omega)}\preceq h^3||v_T||_{H^3(\Omega)}
\end{align*}
\section{Example 6.3}
\section{Example 6.4}

\end{document}